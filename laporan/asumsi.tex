\section{Daftar Asumsi}
\label{sec:asumsi}
Beberapa asumsi yang digunakan dalam pembuatan program DFA ini adalah
\begin{itemize}
  \item Banyak \textit{state} maksimal 1000.
  \item Panjang \textit{state} maksimal 1000 karakter, tanpa spasi.
  \item Simbol berupa karakter bukan spasi.
  \item File eksternal yang digunakan untuk mendapatkan deskripsi DFA bernama \textit{deskripsi.dat} dan berformat sebagai berikut:

  \begin{lstlisting}[frame=single]
(Jumlah state)
(Daftar state, dipisahkan spasi)
(Daftar simbol, tidak dipisahkan spasi)
(State awal)
(Jumlah final state)
(Daftar state akhir, dipisahkan spasi)
(Fungsi transisi berbentuk tabel)
\end{lstlisting}
\end{itemize}
Untuk fungsi transisi, ketentuannya sebagai berikut :
\begin{itemize}
  \item Urutan \textit{state} sesuai penulisan di \textit{deskripsi.dat}
  \item Urutan simbol sesuai penulisan di \textit{deskripsi.dat}
  \item Tabel fungsi transisi terdiri dari sejumlah state baris, yang tiap baris berisi sejumlah simbol state
  \item Untuk setiap $i$ dan $j$ dengan $1 \leq i \leq jumlah-state$ dan $1 \leq j \leq jumlah-simbol$, simbol ke-$j$ akan mengarahkan state $i$ ke state ke-$j$ yang ada di baris ke-$i$
\end{itemize}
