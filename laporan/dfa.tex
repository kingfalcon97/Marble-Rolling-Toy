\section{DFA}
\label{sec:DFA}
Translasi permasalahan \textit{Marble Rolling Toy} dalam notasi DFA :
\begin{equation*}
  A = (Q, \Sigma, \delta, q_0, F)
\end{equation*}
di mana
\begin{align*}
  \centering
  Q &= \{LLLC, RLLC, LRRC, LRLC, RRRC, LRLD, \\&RRLC, LLRD, RRLD, LLLD, RLRD, RLRC, RLLD\} \\
  \Sigma &= \{A, B\} \\
  q_0 &= LLLC \\
  F &= \{LRLD, LLRD, RRLD, LLLD, RLRD, RLLD\} \\
\end{align*}
dan $\delta$ dalam tabel berikut :
\begin{table}[h!]
  \centering
  \caption{Tabel $\delta$ (bukan tabel notasi sederhana DFA)}
  \begin{tabular}{c||c|c}
     $q$ & $\delta(q,A)$ & $\delta(q,B)$ \\
     \hline \hline
     $LLLC$ & $RLLC$ & $LRRC$ \\
     $RLLC$ & $LRLC$ & $RRRC$ \\
     $LRRC$ & $RRRC$ & $LRLD$ \\
     $LRLC$ & $RRLC$ & $LLRD$ \\
     $RRRC$ & $LLRD$ & $RRLD$ \\
     $LRLD$ & $RRLC$ & $LLRD$ \\
     $RRLC$ & $LLLD$ & $RLRD$ \\
     $LLRD$ & $RLRC$ & $LLLD$ \\
     $RRLD$ & $LLLD$ & $RLRD$ \\
     $LLLD$ & $RLLC$ & $LRRC$ \\
     $RLRD$ & $LRRC$ & $RLLD$ \\
     $RLRC$ & $LRRC$ & $RLLD$ \\
     $RLLD$ & $LRLC$ & $RRRC$ \\
  \end{tabular}
\end{table}
